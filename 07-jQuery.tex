\hypertarget{jquery}{%
\section{jQuery}\label{jquery}}

\begin{itemize}
\tightlist
\item
  John Resig, 2006
\item
  Bibliothèque JS, gratuit, OS (licence MIT)
\item
  Facilite le développement JS pour les tâches fréquentes :

  \begin{itemize}
  \tightlist
  \item
    Manipulations DOM
  \item
    Manipulations CSS
  \item
    Réponse aux évenements du navigateur
  \item
    Effets visuels et animations
  \item
    Requêtes et réponses Ajax
  \end{itemize}
\item
  Abstraction implémentations différents navigateurs
\item
  Facile à apprendre
\item
  Utilisation du chaînage des méthodes et des callbacks
\end{itemize}

\hypertarget{utilisation}{%
\section{Utilisation}\label{utilisation}}

\begin{itemize}
\tightlist
\item
  Inclusion \href{http://jquery.com/download/\#other-cdns}{CDN}
\end{itemize}

\begin{otherlanguage}{english}

\begin{Shaded}
\begin{Highlighting}[]
\OperatorTok{<}\NormalTok{script src}\OperatorTok{=}\StringTok{"https://code.jquery.com/jquery-3.1.1.min.js"}\OperatorTok{>}\NormalTok{</script}\OperatorTok{>}
\end{Highlighting}
\end{Shaded}

\end{otherlanguage}

\begin{itemize}
\tightlist
\item
  Nos scripts
\end{itemize}

\begin{otherlanguage}{english}

\begin{Shaded}
\begin{Highlighting}[]
\OperatorTok{<}\NormalTok{script src}\OperatorTok{=}\StringTok{"application.js"}\OperatorTok{>}\NormalTok{</script}\OperatorTok{>}
\end{Highlighting}
\end{Shaded}

\end{otherlanguage}

\begin{itemize}
\tightlist
\item
  Syntaxe basique
\end{itemize}

\begin{otherlanguage}{english}

\begin{Shaded}
\begin{Highlighting}[]
\AttributeTok{$}\NormalTok{(selecteur).}\AttributeTok{action}\NormalTok{()}\OperatorTok{;}      \CommentTok{// $() est un raccourci pour jQuery()}
\end{Highlighting}
\end{Shaded}

\end{otherlanguage}

\begin{itemize}
\tightlist
\item
  Utilisation de sélecteurs CSS, id ou classes
\end{itemize}

\begin{otherlanguage}{english}

\begin{Shaded}
\begin{Highlighting}[]
\AttributeTok{$}\NormalTok{(document)}\OperatorTok{;}                \CommentTok{// retourne le DOM}
\AttributeTok{$}\NormalTok{(}\StringTok{"h3"}\NormalTok{).}\AttributeTok{hide}\NormalTok{()}\OperatorTok{;}             \CommentTok{// cache tous les éléments h3}
\AttributeTok{$}\NormalTok{(}\StringTok{".post"}\NormalTok{)}\OperatorTok{;}                 \CommentTok{// sélectionne les éléments de classe "post"}
\KeywordTok{var}\NormalTok{ node }\OperatorTok{=} \AttributeTok{$}\NormalTok{(}\StringTok{'<p>New</p>'}\NormalTok{)}\OperatorTok{;} \CommentTok{// un nouveau noeud}
\end{Highlighting}
\end{Shaded}

\end{otherlanguage}

\begin{itemize}
\tightlist
\item
  Pour être sûr que le document est chargé :
\end{itemize}

\begin{otherlanguage}{english}

\begin{Shaded}
\begin{Highlighting}[]
\AttributeTok{$}\NormalTok{(document).}\AttributeTok{ready}\NormalTok{(}\KeywordTok{function}\NormalTok{()}\OperatorTok{\{}
    \VariableTok{console}\NormalTok{.}\AttributeTok{log}\NormalTok{(}\StringTok{"prêt!"}\NormalTok{)}
\OperatorTok{\}}\NormalTok{)}\OperatorTok{;}
\end{Highlighting}
\end{Shaded}

\end{otherlanguage}

ou

\begin{otherlanguage}{english}

\begin{Shaded}
\begin{Highlighting}[]
\AttributeTok{$}\NormalTok{(}\KeywordTok{function}\NormalTok{() }\OperatorTok{\{}
    \VariableTok{console}\NormalTok{.}\AttributeTok{log}\NormalTok{(}\StringTok{"prêt!"}\NormalTok{)}
\OperatorTok{\}}\NormalTok{)}\OperatorTok{;}
\end{Highlighting}
\end{Shaded}

\end{otherlanguage}

\begin{otherlanguage}{english}

\end{otherlanguage}

\hypertarget{suxe9lection-dans-le-dom}{%
\section{Sélection dans le DOM}\label{suxe9lection-dans-le-dom}}

\begin{itemize}
\tightlist
\item
  Sélection
\end{itemize}

\begin{otherlanguage}{english}

\begin{Shaded}
\begin{Highlighting}[]
\AttributeTok{$}\NormalTok{(}\StringTok{"h1"}\NormalTok{)}\OperatorTok{;}                        \CommentTok{// noeud élément}
\AttributeTok{$}\NormalTok{(}\StringTok{"h1"}\NormalTok{).}\AttributeTok{text}\NormalTok{()}\OperatorTok{;}                 \CommentTok{// noeud texte en lecture}
\end{Highlighting}
\end{Shaded}

\end{otherlanguage}

\begin{itemize}
\tightlist
\item
  Modification
\end{itemize}

\begin{otherlanguage}{english}

\begin{Shaded}
\begin{Highlighting}[]
\AttributeTok{$}\NormalTok{(}\StringTok{"h1"}\NormalTok{).}\AttributeTok{text}\NormalTok{(}\StringTok{"Nouveau Texte"}\NormalTok{)}\OperatorTok{;} \CommentTok{// noeud texte modifié}
\end{Highlighting}
\end{Shaded}

\end{otherlanguage}

\begin{itemize}
\tightlist
\item
  Tous les fils (sélecteur descendant)
\end{itemize}

\begin{otherlanguage}{english}

\begin{Shaded}
\begin{Highlighting}[]
\AttributeTok{$}\NormalTok{(}\StringTok{"#intro li"}\NormalTok{)}\OperatorTok{;}
\end{Highlighting}
\end{Shaded}

\end{otherlanguage}

\begin{itemize}
\tightlist
\item
  Que les fils directs (sélecteur d'enfants)
\end{itemize}

\begin{otherlanguage}{english}

\begin{Shaded}
\begin{Highlighting}[]
\AttributeTok{$}\NormalTok{(}\StringTok{"#intro > li"}\NormalTok{)}\OperatorTok{;}
\end{Highlighting}
\end{Shaded}

\end{otherlanguage}

\begin{itemize}
\tightlist
\item
  Sélecteur multiple
\end{itemize}

\begin{otherlanguage}{english}

\begin{Shaded}
\begin{Highlighting}[]
\AttributeTok{$}\NormalTok{(}\StringTok{".post, #main "}\NormalTok{)}\OperatorTok{;}
\end{Highlighting}
\end{Shaded}

\end{otherlanguage}

\begin{itemize}
\tightlist
\item
  D'autres
  \href{http://www.w3schools.com/jquery/jquery_selectors.asp}{exemples}
  de sélecteurs
\end{itemize}

\hypertarget{parcours-traversing3}{%
\section{\texorpdfstring{Parcours
(\href{http://www.w3schools.com/jquery/jquery_traversing.asp}{traversing})}{Parcours (traversing)}}\label{parcours-traversing3}}

\begin{itemize}
\tightlist
\item
  Parcours du DOM dans les trois directions :

  \begin{itemize}
  \tightlist
  \item
    Depuis le noeud courant (sélectionné)
  \item
    Haut :
    \begin{otherlanguage}{english}\texttt{parent(),\ parents()}\end{otherlanguage}
  \item
    Bas :
    \begin{otherlanguage}{english}\texttt{children(),\ find()}\end{otherlanguage}
  \item
    Frères :
    \begin{otherlanguage}{english}\texttt{sibling(),\ next(),\ prev()}\end{otherlanguage}
  \end{itemize}
\item
  Filtrage

  \begin{itemize}
  \tightlist
  \item
    \begin{otherlanguage}{english}\texttt{first(),\ last(),\ eq()}\end{otherlanguage}
  \item
    \begin{otherlanguage}{english}\texttt{filter(),\ not()}\end{otherlanguage}
  \item
    \href{http://www.w3schools.com/jquery/jquery_ref_traversing.asp}{Référence}
  \end{itemize}
\end{itemize}

\hypertarget{modifications-de-contenu}{%
\section{Modifications de contenu}\label{modifications-de-contenu}}

\begin{itemize}
\tightlist
\item
  Accès au contenu :

  \begin{itemize}
  \tightlist
  \item
    \begin{otherlanguage}{english}\texttt{text()}\end{otherlanguage} :
    get/set le texte entre les balises
  \item
    \begin{otherlanguage}{english}\texttt{html()}\end{otherlanguage} :
    get/set l'élément complet (yc balises)
  \item
    \begin{otherlanguage}{english}\texttt{val()}\end{otherlanguage} :
    get/set les valeurs d'un formulaire
  \item
    \begin{otherlanguage}{english}\texttt{attr()}\end{otherlanguage} :
    set la valeur d'un attribut
  \end{itemize}
\item
  Ajout de contenu :

  \begin{itemize}
  \tightlist
  \item
    \begin{otherlanguage}{english}\texttt{append(),\ prepend()}\end{otherlanguage}
    : au début/fin de la sélection (dans l'élément)
  \item
    \begin{otherlanguage}{english}\texttt{before(),\ after()}\end{otherlanguage}
    : avant/après la sélection
  \end{itemize}
\item
  Suppression

  \begin{itemize}
  \tightlist
  \item
    \begin{otherlanguage}{english}\texttt{empty()}\end{otherlanguage} :
    suppression des enfants
  \item
    \begin{otherlanguage}{english}\texttt{remove()}\end{otherlanguage} :
    supression de la sélection (possibilité de filtrer)
  \end{itemize}
\end{itemize}

\hypertarget{accuxe8s-aux-css}{%
\section{Accès aux CSS}\label{accuxe8s-aux-css}}

\begin{itemize}
\tightlist
\item
  Accès aux classes

  \begin{itemize}
  \tightlist
  \item
    \begin{otherlanguage}{english}\texttt{addClass()}\end{otherlanguage}
    : ajout de classe(s) à l'élément sélectionné
  \item
    \begin{otherlanguage}{english}\texttt{removeClass()}\end{otherlanguage}
    : suppression de classe(s)
  \item
    \begin{otherlanguage}{english}\texttt{toggleClass()}\end{otherlanguage}
    : suppression si présente, ajout sinon
  \end{itemize}
\item
  Attribut style d'un élément :
  \begin{otherlanguage}{english}\texttt{css()}\end{otherlanguage}
\end{itemize}

\begin{otherlanguage}{english}

\begin{Shaded}
\begin{Highlighting}[]
\AttributeTok{$}\NormalTok{(}\StringTok{"p"}\NormalTok{).}\AttributeTok{css}\NormalTok{(}\StringTok{"background-color"}\NormalTok{)}\OperatorTok{;}                 \CommentTok{// get}
\AttributeTok{$}\NormalTok{(}\StringTok{"p"}\NormalTok{).}\AttributeTok{css}\NormalTok{(}\OperatorTok{\{}\StringTok{"background-color"}\OperatorTok{:}\StringTok{"yellow"}\OperatorTok{,}\StringTok{"font-size"}\OperatorTok{:}\StringTok{"200%"}\OperatorTok{\}}\NormalTok{)}\OperatorTok{;}   \CommentTok{// set}
\end{Highlighting}
\end{Shaded}

\end{otherlanguage}

\hypertarget{evuxe9nements}{%
\section{Evénements}\label{evuxe9nements}}

\begin{itemize}
\tightlist
\item
  Souris

  \begin{itemize}
  \tightlist
  \item
    \begin{otherlanguage}{english}\texttt{click,\ dblclick,\ mouseenter,\ mouseleave}\end{otherlanguage}
  \end{itemize}
\item
  Clavier

  \begin{itemize}
  \tightlist
  \item
    \begin{otherlanguage}{english}\texttt{keypress,\ keyup,\ keydown}\end{otherlanguage}
  \end{itemize}
\item
  Formulaires

  \begin{itemize}
  \tightlist
  \item
    \begin{otherlanguage}{english}\texttt{submit,\ change,\ focus,\ blur}\end{otherlanguage}
  \end{itemize}
\item
  Document

  \begin{itemize}
  \tightlist
  \item
    \begin{otherlanguage}{english}\texttt{ready,\ load,\ resize,\ scroll,\ unload}\end{otherlanguage}
  \end{itemize}
\item
  Exemple
\end{itemize}

\begin{otherlanguage}{english}

\begin{Shaded}
\begin{Highlighting}[]
\AttributeTok{$}\NormalTok{(}\StringTok{"p"}\NormalTok{).}\AttributeTok{click}\NormalTok{(}\KeywordTok{function}\NormalTok{()}\OperatorTok{\{}
  \CommentTok{// code à éxecuter ici}
\OperatorTok{\}}\NormalTok{)}\OperatorTok{;} 
\end{Highlighting}
\end{Shaded}

\end{otherlanguage}

\hypertarget{ajax11}{%
\section{\texorpdfstring{\href{http://www.w3schools.com/jquery/jquery_ajax_load.asp}{AJAX}}{AJAX}}\label{ajax11}}

\begin{itemize}
\tightlist
\item
  \begin{otherlanguage}{english}\texttt{\$(selector).load(URL,\ data,\ callback)}\end{otherlanguage}

  \begin{itemize}
  \tightlist
  \item
    \begin{otherlanguage}{english}\texttt{URL}\end{otherlanguage} :
    Ressource ciblée par la requête
  \item
    \begin{otherlanguage}{english}\texttt{data}\end{otherlanguage} :
    éventuel contenu
  \item
    \begin{otherlanguage}{english}\texttt{callback}\end{otherlanguage} :
    fonction de rappel avec 3 paramètres :

    \begin{itemize}
    \tightlist
    \item
      \begin{otherlanguage}{english}\texttt{responseTxt}\end{otherlanguage}
    \item
      \begin{otherlanguage}{english}\texttt{statusTxt}\end{otherlanguage}
    \item
      \begin{otherlanguage}{english}\texttt{xhr}\end{otherlanguage}
    \end{itemize}
  \end{itemize}
\item
  \begin{otherlanguage}{english}\texttt{\$.get(URL,\ callback)}\end{otherlanguage}
\item
  \begin{otherlanguage}{english}\texttt{\$.post(URL,\ data,\ callback)}\end{otherlanguage}
\end{itemize}

\hypertarget{effets-et-animations}{%
\section{Effets et animations}\label{effets-et-animations}}

\begin{itemize}
\tightlist
\item
  \begin{otherlanguage}{english}\texttt{hide(),\ show(),\ toggle()}\end{otherlanguage}
\item
  \begin{otherlanguage}{english}\texttt{fadeIn(),\ fadeOut(),\ fadeToggle()}\end{otherlanguage}
\item
  \begin{otherlanguage}{english}\texttt{slideDown(),\ slideUp(),\ slideToggle()}\end{otherlanguage}
\item
  \href{http://www.w3schools.com/jquery/jquery_animate.asp}{\begin{otherlanguage}{english}\texttt{animate()}\end{otherlanguage}}
\end{itemize}

\hypertarget{alternatives}{%
\section{Alternatives}\label{alternatives}}

\begin{itemize}
\tightlist
\item
  \emph{jQuery aussi, ça fait vieux}, YBL 17.10.29
\item
  \href{https://gist.github.com/paulirish/12fb951a8b893a454b32}{bling.js}
\item
  API
  \href{https://www.w3schools.com/jsref/met_document_queryselectorall.asp}{queryselectorall()}
  au lieu des getElementsBy\ldots{}
\end{itemize}

\hypertarget{ruxe9fuxe9rences}{%
\section{Références}\label{ruxe9fuxe9rences}}

\begin{itemize}
\tightlist
\item
  Site officiel de \href{http://jquery.com/}{jQuery}
\item
  Tutos \href{http://www.w3schools.com/jquery/}{w3schools}
\item
  Tutos \href{http://try.jquery.com/}{codeschools}
\item
  \href{http://www.javascriptoo.com/sizzle}{SizzleJS} : uniquement les
  sélecteurs
\item
  Comparaison avec \href{http://vanilla-js.com/}{Vanilla JS}
\end{itemize}

\begin{otherlanguage}{english}

\end{otherlanguage}

\begin{otherlanguage}{english}

\end{otherlanguage}

\begin{otherlanguage}{english}

\end{otherlanguage}

\hypertarget{sources}{%
\section{Sources}\label{sources}}
